%!TEX root = cl2-project.tex
\section{Introduction}
\label{sec:introduction}

Identifying depression symptoms is a challenging problem faced by health practitioners. About 25 million adults suffer from symptoms of depression in the United States \cite{NAMI2013}. A lot of these go undetected because often people don't seek medical help when confronted with these symptoms. Symptoms that suggest depression include insomnia, urge to cry, loss of appetite, weight loss/gain and since these can occur due to changes in routine, work demands, depression is often mistaken and causes it to go unreported. Also, so far, the clinical assessments for depression (e.g. The Minnesota Multiphasic Inventory, MMPI) have been based on patient's self accounts of depression associated symptoms and this relies heavily on patients' ability to recognize and report symptoms. Suppression of these symptoms is also an issue among individuals in positions that demand lack of depression (e.g. pilots, army officials). 

Alternative methods for evaluating depression are on the rise, which are more far-reaching than the traditional methods. With the outbreak of the social network, there is data from social networks which contain a trace of online activities. With careful examination of this data, people under higher risk of depression can be identified and counseled before they are stricken by a massive depression episode. Some of these methods include the strategic placement of online apps in sites frequented by large number of people and collecting data about changes in their behavior without too much intrusion. One such app is the Facebook app. This app presents a few questions regarding their daily routine, sleeping and eating habits and then uses that information to predict if the person has a risk of depression. 


Though the app is not that intrusive, it only can predict depression for users that use the app truthfully. Hence, predicting risk of depression using user statuses, change in the choice of words and tone is a more non-intrusive way of predicting depression and is even more far-reaching than the app. To this end, in this work, we examine Facebook statuses of users and employ computational linguistics methods for predicting risk of depression. We use supervised machine learning methods to find language features that correlate heavily with depression risk as predicted by the app to learn language indicators that are suggestive of depression.
We explore variations of topic modeling---\textit{seeded topic models} \cite{jagarlamudi12} for understanding the text. We demonstrate that these linguistic tools help capture depression symptoms and help in predicting risk of depression.

The rest of the paper is organized as follows. We start by reviewing some of the existing work in Section \ref{relatedwork}. In Section \ref{sec:problem} we explain the problem more formally, elaborate on the prediction task. In Section \ref{sec:baseline}, we explain the surface level features in our model. We discuss the dataset and preprocessing steps in \ref{sec:data}. In Sections \ref{sec:lda} and \ref{sec:seededlda}, we explore some variants of \textit{topic modeling}. This is followed by an experimental section \ref{sec:results}, where we present quantitative results and analysis of features used in our models. We finally wrap up by discussing future work and next steps in Section \ref{sec:discussion}.

